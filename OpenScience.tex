\documentclass[pdfpagelabels=false, usepdftitle=false]{beamer}

\usepackage[framenumbers, sectiontitles]{ESE}
% \usepackage{appendixnumberbeamer}

\title{What does Open Science mean for Econometricians?}

% \subtitle{}

% - Use the \inst{?} command only if the authors have different affiliation.
\author{
  Andreas~Alfons
}

% - Use the \inst command only if there are several affiliations.
% - Keep it simple, no one is interested in your street address.
\institute[EUR]{Erasmus School of Economics, Erasmus University Rotterdam}

\date{EI Department Meeting, November~9, 2021}


% new commands
\newcommand{\argmin}{\mathop{\text{arg min}}}
\newcommand{\cor}{\mathop{\text{cor}}}


% \newcommand{\backupbegin}{
%    \newcounter{framenumberappendix}
%    \setcounter{framenumberappendix}{\value{framenumber}}
% }
% \newcommand{\backupend}{
%    \addtocounter{framenumberappendix}{-\value{framenumber}}
%    \addtocounter{framenumber}{\value{framenumberappendix}}
% }


% ---------------------
% begin of presentation
% ---------------------

\begin{document}


\ESEtitleframe

% \begin{frame}{Content}
% \tableofcontents
% %\tableofcontents[pausesections]
% \end{frame}


\begin{frame}{Why Open Science in methodological research?}
\begin{itemize}
  \item We wouldn't expect someone to trust a theoretical result without a
  mathematical proof
  \vfill
  \item Why would we expect someone to trust a table with empirical results
  without the code?
\end{itemize}
\end{frame}


\begin{frame}{FAIR principle}
\alert{F}indable

\bigskip
\alert{A}ccessible

\bigskip
\alert{I}nteroperable

\bigskip
\alert{R}eusable

\vfill

\begin{itemize}
  \arrowitem Introduced for scientific data management, but also applies to
  other domains (e.g., code)
\end{itemize}
\end{frame}


\begin{frame}[t]{Low-hanging fruit}
\begin{itemize}
  \arrowitem \alert{Make sure your research is not behind a paywall}
\end{itemize}

\vfill

Open Access publishing:
\medskip
\begin{itemize}
  \item VSNU agreement with many publishers
  \begin{itemize}
    \arrowitem Corresponding author has to be from a Dutch university
  \end{itemize}
  \smallskip
  \item \sout{Erasmus Open Acces fund} (stopped in June 2021)
  \smallskip
  \item Open Access Journals, e.g.:
  \begin{itemize}
    \item Journal of Machine Learning Research
    \item Journal of Statistical Software
    \item Journal of Data Science, Statistics, and Visualization
  \end{itemize}
  \smallskip
  \item Include budget for Open Access fees in grant applications
\end{itemize}
\end{frame}


\begin{frame}[t]{Low-hanging fruit}
\begin{itemize}
  \arrowitem \alert{Make sure your research is not behind a paywall}
\end{itemize}

\vfill

Make a preprint available, e.g.:
\medskip
\begin{itemize}
  \item arXiv: \url{https://arxiv.org/}
  \smallskip
  \item SSRN: \url{https://www.ssrn.com/}
  \smallskip
  \item RePub: \url{https://repub.eur.nl/}
  \smallskip
  \item ERIM Research Report / TI Discussion Paper
\end{itemize}
\end{frame}


\begin{frame}[t]{Low-hanging fruit}
\begin{itemize}
  \arrowitem \alert{Make the code for your method available}
  (increases citations!)
\end{itemize}

\vfill

Any code is better than no code:
\medskip
\begin{itemize}
  \item It doesn't have to pretty
  \smallskip
  \item It doesn't have to be efficient
  \smallskip
  \item You don't have to provide support
\end{itemize}
\end{frame}


\begin{frame}[t]{Low-hanging fruit}
\begin{itemize}
  \arrowitem \alert{Make the code for your method available}
  (increases citations!)
\end{itemize}

\vfill

Make your code easily findable:
\medskip
\begin{itemize}
  \item Put it on a popular code sharing platform, e.g.:
  \begin{itemize}
    \item GitHub: \url{https://github.com/}
    \item GitLab: \url{https://gitlab.com/}
    \item Bitbucket: \url{https://bitbucket.org/}
  \end{itemize}
  \smallskip
  \smallskip
  \item Put it on the EUR Data Repository (figshare):
  \url{https://datarepository.eur.nl/}
  \begin{itemize}
    \arrowitem You get a DOI, so your code is citable!
  \end{itemize}
\end{itemize}
\end{frame}


\begin{frame}[t]{A step further}
\begin{itemize}
  \arrowitem \alert{Make replication files for your analyses available}
\end{itemize}

\vfill

Scripts that reproduce:
\smallskip
\begin{itemize}
  \item Examples
  \smallskip
  \item Simulation studies or benchmark experiments
  \smallskip
  \item In particular all figures and tables of your paper
\end{itemize}

\vfill

\begin{itemize}
  \arrowitem This requires some effort, but:
  \smallskip
  \begin{itemize}
    \item Several (top) journals require this already (e.g., JASA), and this
    will only become more prevalent
    \smallskip
    \item Grant agencies reward Open Science practices
  \end{itemize}
\end{itemize}
\end{frame}


\begin{frame}[t]{A step further}
\begin{itemize}
  \arrowitem \alert{Make replication files for your analyses available}
\end{itemize}

\vfill

Something to think about: future proofing
\medskip
\begin{itemize}
  \item Package version managers, e.g., \pkg{packrat} in \proglang{R}
  \smallskip
  \item Docker container (virtual machine) containing the computational
  environment and replication files
\end{itemize}
\end{frame}


\begin{frame}[t]{A possible long-term project}
\begin{itemize}
  \arrowitem \alert{Use software that is freely available and open source}
\end{itemize}

\vfill

\begin{itemize}
  \item Using your method should not require an expensive license (e.g,
  \proglang{MATLAB}
  \bigskip
  \item Proprietary software can be vague about important computational details
  of implemented algorithms (e.g., Mplus)
\end{itemize}
\end{frame}


\begin{frame}{Open Science at EUR}
EUR can help you with Open Science:

\vfill

\begin{itemize}
  \item Open Science Coordinator at Erasmus Research Services:
  Antonio Schettino
  \bigskip
  \item Data Steward: Lizette Guzman Ramirez
  \bigskip
  \item Open Science Community Rotterdam:
  \url{https://www.openscience-rotterdam.com/}
\end{itemize}
\end{frame}


\begin{frame}{Big-picture discussion point}
\begin{itemize}
  \arrowitem How can we avoid a replication crisis in methodological research?
  \vfill
  \arrowitem \alert{Should there be registered reports for simulation studies
  and benchmark experiments?}
\end{itemize}
\end{frame}


\end{document}
